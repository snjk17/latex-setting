\documentclass[dvipdfmx, draft]{jsarticle}
% draft: 図の枠のみ表示,コンパイルが速い
\usepackage{IPS-abstract}
\usepackage{graphicx}
\usepackage{amsmath,amssymb,latexsym}
\usepackage{txfonts}
\usepackage{bm}
\pagestyle{empty}

%行間の縮小率
\renewcommand{\baselinestretch}{0.88}

\mastertrue
% これをコメントアウトすると,卒論用になる
% Abstract, etitle, eautherを表示させたいときはコメントアウトを外す

\title{タイトル}
\etitle{} %英語タイトル

\author{計画数理学講座\hspace{2zw}氏名}
\eauthor{} %英語著者名

\abstract{100 words. 概要書く
This is a pen. This is a pen. This is a pen. This is a pen. This is a pen. 
This is a pen. This is a pen. This is a pen. This is a pen. This is a pen. 
This is a pen. This is a pen. This is a pen. This is a pen. This is a pen. 
This is a pen. This is a pen. This is a pen. This is a pen. This is a pen. 
This is a pen. This is a pen. This is a pen. This is a pen. This is a pen. 
This is a pen. This is a pen. This is a pen. This is a pen. This is a pen. 
This is a pen. This is a pen. This is a pen. This is a pen. This is a pen.}

\begin{document}
\maketitle
\thispagestyle{empty}

\section{はじめに}
この論文の流れなどをかく。

\section{大項目タイトル}
\subsection{小項目タイトル}
内容を書きます。詳しくかいてね。\par
これで一マス空き改行。\\
これで空けずに改行。
なにも書かなかったら改行しないよ。

一行開けると一マス空き改行になるよ。

式の書き方。
\begin{equation}
    x + y + z = w
\end{equation}
*をつけると式番号なし
\begin{equation*}
    x + y + z = w
\end{equation*}
好きな番号にもできるよ。
\begin{equation*}
    x + y + z = w \tag{7}
\end{equation*}


\subsection{アルゴリズム3}

従来研究\cite{FujOT08,KitF05,FujDT03,FujDI98}を引用しながら、
研究の背景から本研究の目的、概要をまとめます。

\section{○○問題の定式化○○}
\subsection{準備}
有向グラフ$G=(V,E)$の$V$をn個の頂点の集合、$E$を有向辺の集合と考える。2つの頂点$u,v\subseteq V$間の距離を$d(u,v)$とする。($d(u,u)=0$)。
\subsubsection{定性的クラスとパターンクラス}

問題は数式できっちりと定式化しましょう。

\section{○○主結果○○}

主結果は定理としてまとめると、
すっきりします。

\section{○○数値例○○}

簡単な数値例があると、わかりやすいですね。

\section{おわりに}

まとめを書きましょう。

\begin{thebibliography}{9}
    \bibitem{FujOT08}
    Y.~Fujisaki, Y.~Oishi, and R.~Tempo:
    Mixed Deterministic/Randomized Methods for Fixed Order Controller Design,
    \textit{IEEE Transactions on Automatic Control},
    \textbf{53}-9, 2033/2047 (2008)

    \bibitem{KitF05}
    北村亘, 藤崎泰正:
    有界外乱のもとでの不確かなシステムのミニマックス推定,
    計測自動制御学会論文集,
    \textbf{41}-2, 111/117 (2005)

    \bibitem{FujDT03}
    Y.~Fujisaki, F.~Dabbene, and R.~Tempo:
    Probabilistic Design of LPV Control Systems,
    \textit{Automatica},
    \textbf{39}-8, 1323/1337 (2003)

    \bibitem{FujDI98}
    藤崎, 段, 池田:
    入出力データ配列に基づくシステム表現と制御方式,
    システム制御情報学会論文誌,
    \textbf{11}-11, 630/637 (1998)
\end{thebibliography}

\end{document}